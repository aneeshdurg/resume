%%%%%%%%%%%%%%%%%%%%%%%%%%%%%%%%%%%%%%%%%%%%%%%%%%%%%%%%%%%%%%%%%%%%%%
% LaTeX Template: Curriculum Vitae
%
% Source: http://www.howtotex.com/
% Feel free to distribute this template, but please keep the
% referal to HowToTeX.com.
% Date: July 2011
%%%%%%%%%%%%%%%%%%%%%%%%%%%%%%%%%%%%%%%%%%%%%%%%%%%%%%%%%%%%%%%%%%%%%%

\documentclass[paper=a4,fontsize=10pt]{scrartcl} % KOMA-article class

\usepackage[english]{babel}
\usepackage[utf8x]{inputenc}

\usepackage{boxedminipage}
\usepackage{parcolumns}
\usepackage{multicol}
\usepackage[protrusion=true,expansion=true]{microtype}
\usepackage{amsmath,amsfonts,amsthm}     % Math packages
\usepackage{graphicx}                    % Enable pdflatex
\usepackage[svgnames]{xcolor}            % Colors by their 'svgnames'
\usepackage[margin=0.2in]{geometry}
	\textheight=800px                    % Saving trees ;-)
\usepackage{xhfill}
\usepackage{hyperref}
\usepackage{enumitem}

\frenchspacing              % Better looking spacings after periods
\pagestyle{empty}           % No pagenumbers/headers/footers

%%% Custom sectioning (sectsty package)
%%% ------------------------------------------------------------
\usepackage{sectsty}

\sectionfont{%			            % Change font of \section command
	\usefont{OT1}{phv}{b}{n}%		% bch-b-n: CharterBT-Bold font
	\sectionrule{0pt}{0pt}{0pt}{0pt}}

%%% Macros
%%% ------------------------------------------------------------
\newlength{\spacebox}
\settowidth{\spacebox}{8888888888}			% Box to align text
\newlength{\spacerbox}
\settowidth{\spacerbox}{88888888888888888888888888888888888888888888888888888888888888888888888888888888888888}			% Box to align text
\newcommand{\sepspace}{\vspace*{1em}}		% Vertical space macro

\newcommand{\MyName}[1]{% Name
		\hspace{-1em}\Huge \usefont{OT1}{phv}{b}{n}#1
		\par \normalsize \normalfont }

\newcommand{\MySlogan}[1]{ % Slogan (optional)
		\hspace{-2em}\large \usefont{OT1}{phv}{m}{n}\hfill \textit{#1}
		\par \normalsize \normalfont}

\newcommand{\NewPart}[1]{ \noindent \large \usefont{OT1}{phv}{b}{n}\uppercase{#1} \normalfont \normalsize}

\newcommand{\PersonalEntry}[2]{
		\noindent%\hangindent=2em\hangafter=0 % Indentation
		%\parbox{\spacebox}{        % Box to align text
		\textit{#1}	       % Entry name (birth, address, etc.)
		\newline \color{White}ab\color{Black} #2 \newline}% \par%}    % Entry value

\newcommand{\SkillsEntry}[2]{      % Same as \PersonalEntry
		\noindent % Indentation
		      % Box to align text
		\textit{#1}		       % Entry name (birth, address, etc.)
		\hspace{1.5em} #2 \par}    % Entry value

\newcommand{\EducationEntry}[4]{
		\noindent \textbf{#1}     % Study
			\hfill#2 \par  % Duration
		\noindent \textit{#3} \par        % School
		\noindent\hangafter=0 \small #4 % Description
		\normalsize \par}

\newcommand{\CustomEducationEntry}[4]{
		\noindent \hangindent=2em\hangafter=0 \small #4 \hfill      % Study
		\colorbox{Black}{%
			\parbox{8em}{%
			\hfill\color{White}#2}} \par  % Duration
		%\noindent \textit{#3} \par        % School
		%\noindent\hangindent=2em\hangafter=0 \small #4 % Description
		\normalsize \par}

\newcommand{\ProjectEntry}[4]{
		\noindent \textbf{#1} \hfill  \par    % Study
		%\hfill \url{#2} \par  % Duration
		\noindent \hangindent=2em \hangafter=0 \small #4 \hfill  \par
        \noindent \hangindent=2em\hangafter=0 \small \url{#2} \dotfill #3% Box to align text
		%\parbox{\spacebox}{\textit{#3}}			   % Entry name (birth, address, etc.)
		%\hspace{1.5em} \parbox{\spacerbox}{#4}
        \normalsize \par}    % Entry value

\newcommand{\WorkEntry}[4]{				  % Same as \EducationEntry
		\noindent \textbf{#1} \hfill      % Jobname
		\colorbox{Black}{\color{White}#2} \par  % Duration
		\noindent \textit{#3} \par              % Company
		\noindent\hangindent=2em\hangafter=0 \small #4 % Description
		\normalsize \par}
 \usepackage{vwcol}
%%% Begin Document
%%% ------------------------------------------------------------
\begin{document}
% you can upload a photo and include it here...
%\begin{wrapfigure}{l}{0.5\textwidth}
%	\vspace*{-2em}
%		\includegraphics[width=0.15\textwidth]{photo}
%\end{wrapfigure}

%%% Personal details
%%% ------------------------------------------------------------
%\begin{parcolumns}[nofirstindent,sloppy,rulebetween,colwidths={1=.25\linewidth}]{2}
%\colchunk[1]{
\MyName{Aneesh Durg}
\hspace{-2em}\color{White}a\color{Black}\leaders\hrule height 2.4pt\hskip 10pt plus 1fill\color{White}a\color{Black}
\sepspace

\hspace{-1em}
\begin{minipage}[t]{0.2\textwidth}

\NewPart{Contact}{}\newline
% \PersonalEntry{Phone:}{REDACTED}
\PersonalEntry{Mail:}{aneeshdurg17@gmail.com}
\PersonalEntry{Website:}{\url{aneeshdurg.me}}
\PersonalEntry{Github:}{\url{aneeshdurg}}

\NewPart{Education}{}\newline
\SkillsEntry{
  \small
    University of Illinois at Urbana-Champaign
  \normalsize \newline}{Aug 2015-May 2019}{\sepspace Recieved BS in Computer Science \& Mathematics with \textbf{High Distinction}}{}%{\textbf{Current GPA:} 3.57/4.00\newline
%\small\color{White}ab\color{Black}\textbf{Dean's list:} Fall 15, Fall 16\normalsize}

%\NewPart{Education}{}\newline
%\EducationEntry{University of Illinois at Urbana-Champaign}{\newline 2015-2019}{BS in Computer Science \& \newline Mathematics}{\textbf{Current GPA:}\newline \color{White}ab\color{Black} 3.70/4.00 \newline\textbf{Dean's list:}\newline \color{White}ab\color{Black} Fall 15, Fall 16}

\sepspace
\sepspace
\NewPart{Programming Languages}{} \newline
    \SkillsEntry{}{
      \textsc{C},
      \textsc{Python},
      \textsc{JavaScript},
      \textsc{Bash},
      \textsc{Rust},
      \textsc{Haskell},
      \textsc{Sed},
      \textsc{D},
      \textsc{Go},
      \textsc{C++}
    }

\sepspace
\sepspace
\NewPart{Libraries/ Frameworks}{} \newline
\SkillsEntry{}{
  \textbf{ML/AI:}
  \small
    \textsc{Caffe},
    \textsc{OpenCV},
    \textsc{TensorFlow}
  \normalsize
  \newline\newline
  \textbf{Web:}
  \small
    \textsc{Django},
    \textsc{Tornado},
    \textsc{React},
    \textsc{d3.js},
    \textsc{JQuery}
  \normalsize
  \newline\newline
  \textbf{Other:}
  \small
    \textsc{OpenMP},
    \textsc{MKL},
    \textsc{cuBLAS},
    \textsc{Numpy},
    \textsc{DAAL}
  \normalsize
}

\end{minipage}
\hspace{0.5em}
\vline
%}
%\colchunk[2]{
%\begin{multicols}{2}
\hspace{1em}
\begin{minipage}[t]{0.75\textwidth}

%%% Work experience
%%% ------------------------------------------------------------
\NewPart{Work experience}{}

\EducationEntry{Member of Technical Staff}{Aug 2019-Present}{Qumulo Inc. | Seattle, WA}{
\begin{itemize}
  \item[$\bullet$] Implemented negotiation of SMB 3.1.1
  \item[$\bullet$] Implemented SMB server-side copy
  \item[$\bullet$] Implemented encryption capability to support SMB 3.x
    \begin{itemize}
      \item[$\bullet$] Used the \textbf{OpenSSL} library to perform AES encryption
      \item[$\bullet$] Learnt about best practices surrounding encryption
    \end{itemize}
\end{itemize}
}

\EducationEntry{Course Lead}{Jan 2017-May 2019}{Systems Programming (CS 241) | Urbana, IL}{
\begin{itemize}
  \item[$\bullet$] Development lead for assignments, Lab/Office hours assistant, Honors mentor.
  \item[$\bullet$] Proposed and led initiative to redesign assignments to be more realistic such as:
    \begin{itemize}
      \item[$\bullet$] Redesigned a filesystems assignment by creating a library that redirects filesystem calls, which students use to build a virtual filesystem.
      \item[$\bullet$] Implemented a containers assignment where students build a docker-esque container framework using existing tools.
      %\item[$\bullet$] Prototyping exploratory assignments for the honors section that aim to delve into more advanced topics.
    \end{itemize}
  \item[$\bullet$] Successfully mentored honors student teams to complete systems focused projects exploring areas such as distributed systems, compilers and writing kernel modules.
\end{itemize}
}
\EducationEntry{Software Engineering Intern}{May 2018-Aug 2018}{Qumulo Inc. | Seattle, WA}{
\begin{itemize}
  \item[$\bullet$] Improved testing framework by using \textbf{linux namespaces} and sped up testing time by up to 5x.
  \item[$\bullet$] Designed developed a feature to enable IP failover in \textbf{AWS}.
  \item[$\bullet$] Learnt cloud networking best practices and extensively used the \textbf{AWS API}.
\end{itemize}
}

% \EducationEntry{Machine Learning Intern}{May 2017-Aug 2017}{Intel Corporation | Austin, TX}{
% \begin{itemize}
%   \item Evaluated performance of \textbf{ Intel Movidius Neural Compute stick (NCS)}. Proved a linear increase in speed when using multiple \textbf{NCS} devices.
%     %Developed a web interface to demonstrate performance.
%   %\item Achieved 1.5x speedup on \textbf{CNNs}  (\textbf{GoogLeNet}, \textbf{AlexNet}, \textbf{Age-Gender Net}) by using \textbf{NCS}. Compared against CPU/GPU using \textbf{Caffe}.
%    \item  Made a proof of concept demonstrating potential performance gains by parallelizing NCS convolution.
%   % \item[$\bullet$] Also worked on improving the efficiency of \textbf{libSVM} on intel CPUs as a side project. Achieved 4x speed up by parallelizing via \textbf{OpenMP} and by using \textbf{MKL} BLAS libraries on "Up squared" development board (Apollo Lake SoC).
% \end{itemize}
% }

% \EducationEntry{Software Developer}{May 2016-Dec 2016}{Hacklab Innovations | Bangalore, India}{
% \begin{itemize}
% \item Built AAMI - a reading assistant for the blind and visually impaired. Used technologies such as \textbf{OpenCV}, \textbf{Numpy}, \textbf{tesseract-ocr}, and \textbf{Caffe}.
% \item Developed and optimized a real-time imaging algorithm to find text in images and synthesize audio.
% \end{itemize}
% }

%\EducationEntry{Course Assistant}{Jan 2016-May 2017}{CS 196 @ UIUC [\url{https://courses.engr.illinois.edu/cs19625/}]}{
%\begin{itemize}
%\item Mentored the honors section of Intro to CS.
%\item Helped students get familiar with various algorithmic concepts and various strategies regarding creating and %maintaining a scalable project.
%\item Successfully guided 3 teams of students to complete projects in various areas.
%\end{itemize}
%}

%\sepspace
%\sepspace
%\NewPart{Research experience}{}
%
%\EducationEntry{Networking Researcher}{Aug 2017-Dec 2017}{Independent research project with Prof. Matthew Caesar}{
%\begin{itemize}
%\item Exploring video streaming optimization using \textbf{peer-to-peer} techniques (e.g. optimizing YouTube).
%\end{itemize}
%}

% \EducationEntry{Game Developer \& Research Assistant}{May 2016-May 2017}{Project 415x with Prof. Cary Malkiewich \& Prof. Jenya Sapir}{
% \begin{itemize}
% \item Developed an open source game to teach linear algebra concepts. \url{http://project415x.github.io/}
% %\item Held experimental trials to evaluate effectiveness of the game. Play it at \url{http://p415x.xyz/}
% \end{itemize}
% }


%%% Projects
%%% ------------------------------------------------------------
%\sepspace
\sepspace
\NewPart{Projects}{}

\EducationEntry{What Is a Filesystem?}{Javascript}{\url{http://aneeshdurg.me/what_is_a_filesystem}}{
\begin{itemize}
\item An online interactive book with vizualizations that help students learn about filesystem concepts.
\item Features an implementation of a ext2-esque filesystem with animation to illustrate how a disk would be accessed.
\item Features a terminal simulator which has many standard \texttt{GNU/Linux} utilities built in.%, such as \texttt{cat}, \texttt{ls}, \texttt{mount}, among others.
\end{itemize}
}

\EducationEntry{Pianux}{C}{\url{https://github.com/aneeshdurg/pianux}}{
\begin{itemize}
\item Implemented a piano that plays music written to a linux file-like object.
\item Invented a domain specific language to describe music, supporting syntax for loops and multiple output channels that can be dynamically added and removed.
\end{itemize}
}

\EducationEntry{Algebraic C}{C}{\url{https://github.com/aneeshdurg/algebraic-c}}{
\begin{itemize}
\item Algebraic data types implemented in C.
\item Leverages the \textbf{C preprocessor} to guarantee type safety.
\item Built a custom preprocessor in \textbf{python} to aid code generation.
\end{itemize}
}

%\EducationEntry{Browser Haskell}{Haskell}{\url{https://github.com/aneeshdurg/browserHaskell/}}{
%\begin{itemize}
%\item Online REPL written in \textbf{Haskell} that allows users to run Haskell code.
%\item Replicated terminal experience by leveraging \textbf{websockets} to control standard input.
%\item Used \textbf{Docker} containers to prevent execution of malicious code on the server.
%\end{itemize}
%}

\EducationEntry{CameraTheremin}{JavaScript}{\url{https://aneeshdurg.me/CameraTheremin}}{
\begin{itemize}
  \item Built an in-browser webcam theremin that can play musical notes controlled by hand gestures.
  \item Implemented all image processing functions required.
\end{itemize}
}
%}

\end{minipage}
\end{document}

