%%%%%%%%%%%%%%%%%%%%%%%%%%%%%%%%%%%%%%%%%%%%%%%%%%%%%%%%%%%%%%%%%%%%%%
% LaTeX Template: Curriculum Vitae
%
% Source: http://www.howtotex.com/
% Feel free to distribute this template, but please keep the
% referal to HowToTeX.com.
% Date: July 2011
%%%%%%%%%%%%%%%%%%%%%%%%%%%%%%%%%%%%%%%%%%%%%%%%%%%%%%%%%%%%%%%%%%%%%%

\documentclass[paper=a4,fontsize=11pt]{scrartcl} % KOMA-article class

\usepackage[english]{babel}
\usepackage[utf8x]{inputenc}

\usepackage{boxedminipage}
\usepackage{parcolumns}
\usepackage{multicol}
\usepackage[protrusion=true,expansion=true]{microtype}
\usepackage{amsmath,amsfonts,amsthm}     % Math packages
\usepackage{graphicx}                    % Enable pdflatex
\usepackage[svgnames]{xcolor}            % Colors by their 'svgnames'
\usepackage[margin=0.25in]{geometry}
	\textheight=800px                    % Saving trees ;-)
\usepackage{xhfill}
\usepackage{hyperref}
\usepackage{enumitem}
\usepackage{vwcol}

\frenchspacing              % Better looking spacings after periods
\pagestyle{empty}           % No pagenumbers/headers/footers

%%% Custom sectioning (sectsty package)
%%% ------------------------------------------------------------
\usepackage{sectsty}

\sectionfont{%			            % Change font of \section command
	\usefont{OT1}{phv}{b}{n}%		% bch-b-n: CharterBT-Bold font
	\sectionrule{0pt}{0pt}{0pt}{0pt}}

%%% Macros
%%% ------------------------------------------------------------
\newlength{\spacebox}
\settowidth{\spacebox}{8888888888}			% Box to align text
\newlength{\spacerbox}
\settowidth{\spacerbox}{88888888888888888888888888888888888888888888888888888888888888888888888888888888888888}			% Box to align text
\newcommand{\sepspace}{\vspace*{1em}}		% Vertical space macro

\newcommand{\MyName}[1]{% Name
		\hspace{-1em}\Huge \usefont{OT1}{phv}{b}{n}#1
		\par \normalsize \normalfont }

\newcommand{\MySlogan}[1]{ % Slogan (optional)
		\hspace{-2em}\large \usefont{OT1}{phv}{m}{n}\hfill \textit{#1}
		\par \normalsize \normalfont}

\newcommand{\NewPart}[1]{ \noindent \large \usefont{OT1}{phv}{b}{n}\uppercase{#1} \normalfont \normalsize}

\newcommand{\EducationEntry}[4]{
		\noindent \textbf{#1}     % Study
			\hfill#2 \par  % Duration
		\noindent \textit{#3} \par        % School
		\noindent\hangafter=0 \small #4 % Description
		\normalsize \par}
%%% Begin Document
%%% ------------------------------------------------------------
\begin{document}

\MyName{Aneesh Durg}
\sepspace
Email: aneeshdurg17@gmail.com $\vert$ Website: \url{https://aneeshdurg.me} $\vert$ Github: \url{https://github.com/aneeshdurg}
\newline
\vspace{-1em}

\hspace{-2em}\color{White}a\color{Black}\leaders\hrule height 2.4pt\hskip 10pt plus 1fill\color{White}a\color{Black}

\NewPart{Education}{}

\EducationEntry{BS in Computer Science \& Mathematics}{Aug 2015 - May 2019}{University of Illinois at Urbana-Champaign} {
\begin{itemize}
  \item GPA: 3.57/4.00
  \item Graduated with High Distinction
  \item Was included in the Dean's List in Fall '15 and Fall '16
\end{itemize}

}

\NewPart{Work experience}{}

\EducationEntry{Senior Software Engineer}{Jul 2023-Present}{Bodo.ai | remote}{
\begin{itemize}
  \item Developing the core engine which consists of an optimizing compiler and scalable runtime for \textbf{SQL} and python/pandas workflows.
  \item Expanding compiler and runtime support for non-ANSI SQL dialects
  \item Identified optimizations that reduced compile time by 60\% in some benchmarks
\end{itemize}
}

\EducationEntry{Senior Software Engineer/Team Lead}{Feb 2021-Present}{KatanaGraph Inc. | Austin, TX}{
\begin{itemize}
  \item Worked on building a distributed graph compute engine that provides AI, analytics, and a database.
  \item Lead a team of \textbf{5} to implement and support graph database querying and ingest.
  \begin{itemize}
    \item[$\bullet$] Guided design discussions, identified organizational blockers, and coordinated with product to set priorities and generate new technical requirements.
  \end{itemize}
  \item Implemented compiler and runtime support for the \textbf{Cypher} query language.
  \item Designed and implemented novel high performance algorithms for distributed subgraph pattern matching (tested on $\sim$\textbf{20B} nodes, \textbf{44B} edges)
  \begin{itemize}
    \item[$\bullet$] Improved performance by \textbf{100x} in queries against the \textbf{LDBC-SNB} datasets and reduced memory usage by over \textbf{95\%} on benchmarks simulating specific client workloads.
  \end{itemize}
  \item Proposed and implemented AST transformations to optimize query performance
  \item Designed syntax extensions to \textbf{Cypher} to allow users to tune query performance
  \item Built a hotswap mechanism to allow devs to replace only salient parts of a katana deployment on \textbf{kubernetes}, reducing org-wide feedback cycles by up to \textbf{30x}
  \item Built infrastructure for benchmarking the query engine in isolation from the rest of the product using \textbf{slurm}
\end{itemize}
}

\EducationEntry{Member of Technical Staff}{Aug 2019-Feb 2021}{Qumulo Inc. | Seattle, WA}{
\begin{itemize}
  \item Extended platform support for two new hardware configurations
  \item Designed a solution for eliminating server downtime during upgrades from 5 minutes to under 30 seconds in a team of four
    \begin{itemize}
        \item Used containerization to avoid potentially slow boot times
        \item Used a \textbf{dbus}-based mechanism to allow processes to break out of the container and control the host
    \end{itemize}
  \item Implemented \textbf{SMB} server-side copy and \textbf{SMB}'s encryption protocol
    % highlight the problem solved
  \item Refactored the \textbf{SMB} implementation to reduce memory usage and make object lifetimes more explicit.
  \item Helped lead migration of \textbf{python2} code to \textbf{python3}
    \begin{itemize}
      \item Modernized python code by adding types via \textbf{mypy}
      \item Proposed and implemented a python dependency verification tool for customer and cloud deployments
    \end{itemize}
\end{itemize}
}

\EducationEntry{Software Engineering Intern}{May 2018-Aug 2018}{Qumulo Inc. | Seattle, WA}{
\begin{itemize}
  \item Worked on migrating an on-prem filesystem to work in AWS
  \item Helped implement a new hardware abstraction layer to interact with AWS resources
  \item Designed and developed an IP failover solution in \textbf{AWS}.
  \item Used \textbf{linux namespaces} to speed up testing time by up to 5x.
\end{itemize}
}

\EducationEntry{Machine Learning Intern}{May 2017-Aug 2017}{Intel Corporation | Austin, TX}{
\begin{itemize}
  \item Evaluated performance of \textbf{ Intel Movidius Neural Compute stick (NCS)}.
  \item Proposed and built a tool to split large networks across multiple \textbf{NCS} devices
  \item Developed a browser plugin to demonstrate real-time image recognition on Raspberry PIs using \textbf{NCS}.
  \item Developed a benchmarking suite to demonstrate a 1.5x speedup on \textbf{CNNs}  (\textbf{GoogLeNet}, \textbf{AlexNet}, \textbf{Age-Gender Net}) by using \textbf{NCS}. Compared against CPU/GPU using \textbf{Caffe}.
  %\item  Made a proof of concept demonstrating potential performance gains by parallelizing NCS convolution.
  \item Improved performance of \textbf{libSVM} on intel CPUs by using \textbf{OpenMP} for parallelism and \textbf{MKL} BLAS libraries to use intel CPU specific BLAS instructions. Achieved a 4x speed on the "Up squared" development board (Apollo Lake SoC).
\end{itemize}
}

\EducationEntry{Software Developer}{May 2016-Dec 2016}{Hacklab Innovations | Bangalore, India}{
\begin{itemize}
\item Built AAMI - a wearable reading assistant for the blind and visually impaired.
\item Developed and optimized a real-time imaging solution to find text in images and synthesize audio using \textbf{OpenCV}, \textbf{tesseract-ocr}, and \textbf{Caffe}.
\item Designed and built a tactile feedback mechanism to help visually impaired users navigate lines of text.
\end{itemize}
}


\sepspace
\NewPart{Teaching Experience/Projects}{}
\newline

\EducationEntry{What Is a Filesystem?}{}{\url{http://aneeshdurg.me/what_is_a_filesystem}}{
\begin{itemize}
\item An interactive book/vizualization for students learning filesystem concepts.
\item Implements a ext2-esque filesystem with animations to illustrate how a disk accesses occur.
\item Features a terminal simulator which implements some standard \texttt{GNU/Linux} utilities and interactively visualizes how they interact with filesystems and disk IO.%, such as \texttt{cat}, \texttt{ls}, \texttt{mount}, among others.
\end{itemize}
}

\EducationEntry{Visual Malloc}{}{\url{https://aneeshdurg.me/visual-malloc/}} {
\begin{itemize}
\item An interactive vizualization to aid in teaching students about how memory allocators work, and possibly to allow students to use as a debugging tool when implementing their own mallocs.
\end{itemize}
}

\EducationEntry{Systems Programming Course Lead}{Jan 2017-May 2019}{CS241 @ UIUC | Urbana, IL}{
\begin{itemize}
  \item Development lead for assignments, Lab/Office hours assistant, Honors mentor.
  \item Designed and created assignments (and associated infrastructure) to allow students to implement and explore concepts such as filesystems, containers, and cooperative scheduling.
  \item Mentored honors students to complete projects exploring areas such as distributed systems, compilers and linux kernel development.
  \item Wrote and gave lectures on additional topics such as containerization, and kernel developement for the honors section
  \item Held review sessions for assignments with low average score by creating slides and handouts that demonstrated concepts through hands-on guided exploration of topics
\end{itemize}
}

\EducationEntry{Illinois-CS241 Coursebook}{}{\url{https://github.com/illinois-cs241/coursebook}}{
\begin{itemize}
  \item Helped write and review portions of the free coursebook, which covers a superset of all content from UIUC's CS241
  \item Contributed chapters on filesystems, containers, and basic kernel developement.
\end{itemize}
}

\EducationEntry{Research Game Developer}{May 2016-May 2017}{Project 415x @ UIUC with Prof. Cary Malkiewich \& Prof. Jenya Sapir}{
\begin{itemize}
\item \url{https://github.com/project415x/project415x.github.io}
\item Developed an open source game to kinesthetically teach linear algebra concepts.
\item Held experimental trials to evaluate effectiveness of the game, but the results were inconclusive.
\end{itemize}
}

\sepspace
\NewPart{Projects}{}
\newline
\EducationEntry{rainbow}{python/Cypher}{\url{https://github.com/aneeshdurg/rainbow}}{
\begin{itemize}
  \item Arbitrary compile-time function coloring and callgraph rejection tool powered by \textbf{clang} and \textbf{Cypher}
\item Provides an ergonomic way for users to labels functions and lambdas, and then define relationships between those labels that should be considered invalid. Some example usecases are:
  \begin{itemize}
    \item[$\bullet$] label functions that assume locks are held to verify that they are never called without a lock
    \item[$\bullet$] label routines using collective MPI operations to ensure that other collective operations aren't called during execution
    \item[$\bullet$] prototype new language features such as async/constexpr without writing custom compiler passes/extensions
  \end{itemize}
\end{itemize}
}

\EducationEntry{spycy}{python/WASM}{\url{https://github.com/aneeshdurg/spycy}}{
\begin{itemize}
  \item An in-process graph database library for python that implements a \textbf{openCypher} frontend
  \item Provides implementable interfaces for data sources to enable querying real world graphs.
  \begin{itemize}
    \item[$\bullet$] Wrote a demo that uses \textbf{spycy} and \textbf{WASM} to filter HTML nodes using \textbf{openCypher}
  \end{itemize}
\end{itemize}
}

\EducationEntry{Bash Raytracer}{bash}{\url{https://github.com/aneeshdurg/bash-raytracer}}{
\begin{itemize}
\item An implementation of a raytracer in bash
\item Inspired by the CMake raytracer, this project aims to use bash implement a raytracer that uses fixed point arithmetic. The purpose was to test my bash skills and learn about raytracing.
\end{itemize}
}

\EducationEntry{Video Synthesizer}{Javascript/GLSL}{\url{https://aneeshdurg.me/video-synth}}{
\begin{itemize}
\item A GPU accelerated interface to build complex generative visual effects that achieve real-time manipulation of audio and video input.
\item Features modules that can be chained and combined with various operators
\end{itemize}
}

\EducationEntry{SignalApps}{Rust/Python}{\url{https://github.com/aneeshdurg/signalapps}} {
\begin{itemize}
  \item A platform to build secure and anonymized chatbot based applications on top of the \textbf{Signal} protocol
\end{itemize}

}

\EducationEntry{Ephemeral}{Typescript/React}{\url{http://aneeshdurg.me/ephemeral}}{
\begin{itemize}
\item A distributed, peer-to-peer, twitter-inspired social media platform
\item Uses \textbf{RSA} encryption and signing to prove identity
\end{itemize}
}

\EducationEntry{CameraTheremin}{JavaScript/GLSL}{\url{https://aneeshdurg.me/CameraTheremin}}{
\begin{itemize}
\item In-browser webcam gesture-based theremin (a musical instrument)
\item Implemented all image processing functions required in Javascript and again in \textbf{WebGL} to compare performance.
\end{itemize}
}

\end{document}

